\documentclass{article}

% Language setting
% Replace `english' with e.g. `spanish' to change the document language
%\usepackage[english]{babel}
\renewcommand{\figurename}{Figura}
\newcommand{\R}{\mathbb{R}}

% Set page size and margins
% Replace `letterpaper' with`a4paper' for UK/EU standard size
\usepackage[letterpaper,top=2cm,bottom=2cm,left=3cm,right=3cm,marginparwidth=1.75cm]{geometry}
\renewcommand{\figurename}{Figura}

% Useful packages
\usepackage{amsmath}
\usepackage{graphicx}
\usepackage{amssymb}
\usepackage{amsfonts}
\usepackage{amstext}
\usepackage{amsthm}
\usepackage{csquotes}
\usepackage[colorlinks=true, allcolors=blue]{hyperref}

\title{Trabalho Complementar 06}
\author{Bruno Rafael dos Santos}
\date{05 de Abril de 2024}

\begin{document}
\maketitle

\section*{6.4) Em um sensor, o que é a função transferência?}
É uma função que tem como argumento o estímulo físico (recebido pelo sensor) e como resultado o sinal (emitido pelo sensor).


\section*{6.6) Quase todos os atuadores podem ser classificados em 
três categorias, segundo o tipo de energia utilizada. Defina essas categorias?}
Atuadores são dispositivos de hardware que convertem sinais de comando em alterações de parâmetros físicos.

As categorias de atuadores são:
    \begin{itemize}
        \item \textbf{Elétricos}: são os mais comuns (ex.: motores elétricos) podendo ser lineares (em que a saída é um deslocamento linear) ou angulares (em que a saída é um deslocamento angular).
        \item \textbf{Hidráulicos}: usam fluído (líquido) para amplificar a alteração de parâmetro físico realizada, por isso costumam ser utilizado quando se requer força mais elevada, e, além disso, podem realizar movimento linear ou rotativo.
        \item \textbf{Pneumáticos}: são semelhantes aos hidráulicos, porém utilizam fluídos gasosos, e, além disso, também podem realizar movimento linear ou rotativo, mas diferentemente dos atuadores hidráulicos costumam estar mais presentes em aplicações que requerem relativamente pouca força.
    \end{itemize}


\end{document}